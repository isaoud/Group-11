% Compile with LuaLaTeX or XeLaTeX!
% %%%%%%%%%%%%%%%%%%%%%%%%%%%%%%%%%%%%%%%%%%%%%%%%%%%%%%%%%%%%%%%%%%%%%%%%%%%% %
%                          Begin   Header                                      %
% %%%%%%%%%%%%%%%%%%%%%%%%%%%%%%%%%%%%%%%%%%%%%%%%%%%%%%%%%%%%%%%%%%%%%%%%%%%% %
\documentclass[a4paper,11pt]{article}             % bestimmt das Aussehen eines Dokuments

\usepackage[ngerman]{babel}         % Neue deutsche Rechtschreibung
\usepackage[T1]{fontenc}            % Bessere Schriftdarstellung
\usepackage{lmodern}                % Aktuelle Schrift

\usepackage[intlimits]{amsmath}     % Zusaetzliche Matheumgebungen
\usepackage{amssymb}                % Mathematische Symbole
\usepackage{graphicx}               % notwendig fuer \includegraphics
\usepackage{fancyhdr}               % Kopf- und Fusszeilehttps://www.overleaf.com/9784728wrqtfwfjtvmz#
\usepackage{lastpage}               % erzeugt Referez zu der letzten Seite
\usepackage{moreverb}               % verbatimtab Umgeung
\usepackage{tikz}
\usetikzlibrary{decorations.pathmorphing,shapes,decorations.text,decorations,calc,positioning,automata,matrix,arrows}
\usepackage{fontspec}

% Java Code Formatting
\usepackage{fontspec}
\setmonofont[AutoFakeBold=1.5, AutoFakeSlant=0.4]{Inconsolata}
\usepackage{listings}
\usepackage{color}

\definecolor{dkgreen}{rgb}{0,0.6,0}
\definecolor{gray}{rgb}{0.5,0.5,0.5}
\definecolor{mauve}{rgb}{0.58,0,0.82}
% \setmonofont[
%   AutoFakeSlant,
%   BoldItalicFeatures={FakeSlant},
% ]{Inconsolata}

% {
% Extension = .ttf ,
% UprightFont = CharisSILR,
% BoldFont = CharisSILB,
% ItalicFont = CharisSILI,
% BoldItalicFont = CharisSILBI,
% % <any other desired options>

% \newfontfamily{\consolas}{consolas}[Extension = .ttf, UprightFont = *-regular, ItalicFont = *-italic, BoldItalicFont = *-z, BoldFont = *-bold]
% \setmonofont[UprightFont={[consolas-regular.ttf]}, ItalicFont={[consolas-italic.ttf]}]{Consolas}

% Seiteneinstelungen
\setlength\textwidth{165mm}           % Breite
\setlength\textheight{235mm}          % Hoehe
\setlength\headheight{41pt}           % Hoehe der Kopfzeile
\setlength\topmargin{-12mm}           % Abstand oben
\setlength\oddsidemargin{0mm}         % Linker Rand
\setlength\parindent{0pt}             % und ohne Einrueckung
\setlength\parskip{1.7\medskipamount} % Absaetze abgesetzt
\sloppy\pagestyle{fancy}

%Kopf- und Fusszeileeinstellungen
\renewcommand{\headrulewidth}{0.4pt} 	%obere Trennlinie
\fancyfoot[C]{Seite:~\thepage~von~\pageref{LastPage}} %Seitennummer
\renewcommand{\footrulewidth}{0.4pt} 	%untere Trennlinie

% Ein paar hilfreiche Zeichen
\newcommand{\R}{\mathbb{R}}                 % reelle Zahlen
\newcommand{\N}{\mathbb{N}}                 % natuerliche Zahlen
\newcommand{\e}{\text{e}}                   % eulersche Zahl
\newcommand{\E}[1]{\cdot10^{#1}}            % x 10^{...}
\newcommand{\qed}{\hspace*{\fill}q.e.d.}    % Beweis fertig
\newcommand{\ON}[1]{{\cal O}(#1)}	    %O-Notation
%\newcommand{\Aufgabe}[1]{{\vskip5mm\bf Aufgabe #1.\\}}

\def\vblatt{~}
\def\vtermin{~}
\def\vbriefkasten{~}

%Blattnummer, Abgabedatum und Namen mir Matrikelnummern in der Kopfzeile anpassen
\newcommand{\dreinamen}[6]{\fancyhead[R]{#1 (#2)\\#3 (#4)\\#5 (#6)}} %Kopfzeile rechts
\fancyhead[C]{\large{\bf{Blatt \vblatt }}}
\fancyhead[L]{\textbf{Datenstrukturen und Algorithmen}\\ Sommersemester 2017 \\ \"Ubungsgruppe: \vtermin%\\Briefkasten: \vbriefkasten 
}

\newcommand{\blatt}[1]{\def\vblatt{#1}}
\newcommand{\termin}[1]{\def\vtermin{#1}}
\newcommand{\briefkasten}[1]{\def\vbriefkasten{#1}}

%!!!!Neu Einfaches einbinden von Quelltexten
\usepackage{listings}

%Algorithmennotation als Quelltext verwenden.
\lstdefinelanguage{alglang}
{keywords={if, then, endif, else, repeat, until, endrepeat, while, do, endwhile, var, module, endmodule, mod, return, pre, post, reads, changes, mem, Type, Functions, Preconditions, Axioms},
emph={Integer, Bool, Real, Char, String},
sensitive=false,
morecomment=[l]{--},
morecomment=[s]{\{}{\}},
morestring=[d]"  
}
\lstdefinestyle{algstyle}{
  mathescape=true, 
  basicstyle=\normalfont\sffamily, 
  commentstyle=\normalfont\sffamily, 
  keywordstyle=\sffamily\bfseries,
  identifierstyle=\sffamily\itshape, 
  emphstyle=\normalfont\sffamily, 
  stringstyle=,
  showstringspaces=false,        
  columns=[l]fullflexible     
}

\lstdefinestyle{javastyle}{
	basicstyle=\ttfamily, 
  	commentstyle=\ttfamily, 
  	keywordstyle=\ttfamily,
  	identifierstyle=\ttfamily, 
  	emphstyle=\ttfamily,
    frame=tb,
    language=Java,
    aboveskip=3mm,
    belowskip=3mm,
    showstringspaces=false,
    columns=flexible,
    basicstyle={\small\ttfamily},
    numbers=none,
    numberstyle=\tiny\color{gray},
    keywordstyle=\color{blue},
    commentstyle=\color{dkgreen},
    stringstyle=\color{mauve},
    breaklines=true,
    breakatwhitespace=true,
    tabsize=3
}

\newcommand{\lstjava}{\lstset { numbers=left,language=java,tabsize=2,numberstyle=\tiny ,numbersep=5pt,basicstyle=\scriptsize}}
\newcommand{\lstalg}{\lstset { numbers=left,language=alglang,tabsize=2,numberstyle=\tiny,style=algstyle}}

\newcounter{aufgabe}
\setcounter{aufgabe}{0}

\newcommand{\Aufgabe}{\noindent\newline\addtocounter{aufgabe}{1}\textbf{Aufgabe \vblatt.\theaufgabe}\\
%\input{\theaufgabe}\bigskip
}

% Unteraufgaben (mit Enumeration)
\def\labelenumi{(\arabic{enumi})}
\parindent0mm % keine Absatzeinrückung

% %%%%%%%%%%%%%%%%%%%%%%%%%%%%%%%%%%%%%%%%%%%%%%%%%%%%%%%%%%%%%%%%%%%%%%%%%%%% %
%                           Ende Header                                        %
% %%%%%%%%%%%%%%%%%%%%%%%%%%%%%%%%%%%%%%%%%%%%%%%%%%%%%%%%%%%%%%%%%%%%%%%%%%%% %

%!!!!anpassen an das Betriebssystem!!!, um Umlaute zu verwenden
% \usepackage[utf8]{inputenc}                      %Linux
%\usepackage[latin1]{inputenc}                    %Windows
%\usepackage[applemac]{inputenc}                  %Mac


%Namen und Matrikelnummern anpassen
\dreinamen{Ibraheem Saoud}{3041390}{Liang Chun Wong}{3037494}{Miroslav Valov}{3023792} %3er Gruppe


%Termin der Uebungsgruppe und Raum anpassen z. B. \termin{Mo 10:15-11:45, BB 915}
\termin{Di 16:15-18:45, BB 915}

%Blattnummer anpassen z. B. \blatt{1}
\blatt{7}


\begin{document}

\Aufgabe
\begin{enumerate}
\item Types
\subitem MAP[K,V]
\item Functions
\subitem mt\_map : MAP[K,V]
\subitem insert : MAP[K,V] $\times$ K $\times$ V $\nrightarrow$ MAP[K,V]
\subitem update : MAP[K,V] $\times$ K $\times$ V $\nrightarrow$ MAP[K,V]
\subitem is\_in\_dom : MAP[K,V] $\times$ K $\rightarrow$ Bool
\subitem lookup : MAP[K,V] $\times$ K $\rightarrow$ V
\subitem lookup\_opt : MAP[K,V] $\times$ K $\rightarrow$ Option[V]
\subitem delete : MAP[K,V] $\times$ K $\nrightarrow$ MAP[K,V]
\subitem union : MAP[K,V] $\times$ MAP[K,V] $\nrightarrow$ MAP[K,V]
\subitem size : MAP[K,V] $\rightarrow$ Integer
% \item Preconditions
% \subitem $\forall$ m : MAP[K,V]; k : K; v : V $\bullet$ 
% \subsubitem is\_in\_dom(m, k) \leftrightarrow lookup(m, k)
% \subsubitem insert(m, k) \leftrightarrow $\not$ lookup
\subitem 
\end{enumerate}

\Aufgabe
\begin{lstlisting}[style=javastyle]
// Map.java 
/**
 * Created by liangchun on 05.06.17.
 */
public interface Map<K, V> {
    void insert(K x, V y);
    void update(K x, V y);
    boolean is_in_dom(K x);
    V lookup(K x);
    MyOption<V> lookup_opt(K x);
    void delete(K x);
    void union(Map<K, V> x);
    int size();
}
\end{lstlisting}
\newpage
\begin{lstlisting}[style=javastyle]
// MyMap.java 
import java.util.HashMap;
/**
 * Created by liangchun on 05.06.17.
 */
public class MyMap<K, V> implements Map<K, V> {
    public HashMap<K, V> map;
    MyMap() {
        this.map = new HashMap<>();
    }
    public void insert(K x, V y) {
        if (this.map.get(x) == null)
            this.map.put(x, y);
        else
            this.update(x, y);
    }
    public void update(K x, V y) {
        if (this.map.get(x) != null)
            this.map.put(x, y);
    }
    public V lookup(K x) {
        return this.map.get(x);
    }
    public void delete(K x) {
        if (this.map.get(x) != null)
            this.map.remove(x);
    }
    public void union(Map<K, V> x) {
        MyMap<K, V> temp = (MyMap<K, V>) x;
        temp.map.putAll(this.map);
        this.map = temp.map;
    }
    public boolean is_in_dom(K x) {
        return (this.map.containsKey(x)) ? true : false;
    }
    public int size() {
        return this.map.size();
    }
    public MyOption<V> lookup_opt(K x) {
        if (this.map.containsKey(x))
            return new MyOptionSome<>(this.map.get(x));
        return new MyOptionNone<>();
    }
}



\end{lstlisting}
\newpage
\begin{lstlisting}[style=javastyle]
/**
 * Created by liangchun on 05.06.17.
 */
public class Main {
    public static void main(String args[]) {
        MyMap<String, Integer> x = new MyMap<>();

        // insert
        x.insert("s", 2);
        System.out.println("Lookup s in x: " + x.lookup("s"));

        // delete
        x.insert("delete", 123);
        x.delete("delete");
        System.out.println("Deleted key-value pair 'delete' from map");

        // lookup
        System.out.println("Lookup 'delete' in x: " + x.lookup("delete"));

        // is_in_dom
        System.out.println("Check if delete is_in_dom in x: " + x.is_in_dom("delete"));

        // update
        x.update("s", 999);
        System.out.println("Update s to 999: " + x.map);

        // union
        MyMap<String, Integer> y = new MyMap<>();
        y.insert("Key", 24);
        y.union(x);
        System.out.println("Union map of y with x: " + y.map);

        // size
        System.out.println("Size of x: " + x.size());
    }
}

\end{lstlisting}


\end{document}
