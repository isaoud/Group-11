% Compile with LuaLaTeX or XeLaTeX!
% %%%%%%%%%%%%%%%%%%%%%%%%%%%%%%%%%%%%%%%%%%%%%%%%%%%%%%%%%%%%%%%%%%%%%%%%%%%% %
%                          Begin   Header                                      %
% %%%%%%%%%%%%%%%%%%%%%%%%%%%%%%%%%%%%%%%%%%%%%%%%%%%%%%%%%%%%%%%%%%%%%%%%%%%% %
\documentclass[a4paper,11pt]{article}             % bestimmt das Aussehen eines Dokuments

\usepackage[ngerman]{babel}         % Neue deutsche Rechtschreibung
\usepackage[T1]{fontenc}            % Bessere Schriftdarstellung
\usepackage{lmodern}                % Aktuelle Schrift

\usepackage[intlimits]{amsmath}     % Zusaetzliche Matheumgebungen
\usepackage{amssymb}                % Mathematische Symbole
\usepackage{graphicx}               % notwendig fuer \includegraphics
\usepackage{fancyhdr}               % Kopf- und Fusszeilehttps://www.overleaf.com/9784728wrqtfwfjtvmz#
\usepackage{lastpage}               % erzeugt Referez zu der letzten Seite
\usepackage{moreverb}               % verbatimtab Umgeung
\usepackage{tikz}
\usetikzlibrary{decorations.pathmorphing,shapes,decorations.text,decorations,calc,positioning,automata,matrix,arrows}
\usepackage{fontspec}

% Java Code Formatting
\usepackage{fontspec}
\setmonofont[AutoFakeBold=1.5, AutoFakeSlant=0.4]{Inconsolata}
\usepackage{listings}
\usepackage{color}

\definecolor{dkgreen}{rgb}{0,0.6,0}
\definecolor{gray}{rgb}{0.5,0.5,0.5}
\definecolor{mauve}{rgb}{0.58,0,0.82}
% \setmonofont[
%   AutoFakeSlant,
%   BoldItalicFeatures={FakeSlant},
% ]{Inconsolata}

% {
% Extension = .ttf ,
% UprightFont = CharisSILR,
% BoldFont = CharisSILB,
% ItalicFont = CharisSILI,
% BoldItalicFont = CharisSILBI,
% % <any other desired options>

% \newfontfamily{\consolas}{consolas}[Extension = .ttf, UprightFont = *-regular, ItalicFont = *-italic, BoldItalicFont = *-z, BoldFont = *-bold]
% \setmonofont[UprightFont={[consolas-regular.ttf]}, ItalicFont={[consolas-italic.ttf]}]{Consolas}

% Seiteneinstelungen
\setlength\textwidth{165mm}           % Breite
\setlength\textheight{235mm}          % Hoehe
\setlength\headheight{41pt}           % Hoehe der Kopfzeile
\setlength\topmargin{-12mm}           % Abstand oben
\setlength\oddsidemargin{0mm}         % Linker Rand
\setlength\parindent{0pt}             % und ohne Einrueckung
\setlength\parskip{1.7\medskipamount} % Absaetze abgesetzt
\sloppy\pagestyle{fancy}

%Kopf- und Fusszeileeinstellungen
\renewcommand{\headrulewidth}{0.4pt} 	%obere Trennlinie
\fancyfoot[C]{Seite:~\thepage~von~\pageref{LastPage}} %Seitennummer
\renewcommand{\footrulewidth}{0.4pt} 	%untere Trennlinie

% Ein paar hilfreiche Zeichen
\newcommand{\R}{\mathbb{R}}                 % reelle Zahlen
\newcommand{\N}{\mathbb{N}}                 % natuerliche Zahlen
\newcommand{\e}{\text{e}}                   % eulersche Zahl
\newcommand{\E}[1]{\cdot10^{#1}}            % x 10^{...}
\newcommand{\qed}{\hspace*{\fill}q.e.d.}    % Beweis fertig
\newcommand{\ON}[1]{{\cal O}(#1)}	    %O-Notation
%\newcommand{\Aufgabe}[1]{{\vskip5mm\bf Aufgabe #1.\\}}

\def\vblatt{~}
\def\vtermin{~}
\def\vbriefkasten{~}

%Blattnummer, Abgabedatum und Namen mir Matrikelnummern in der Kopfzeile anpassen
\newcommand{\dreinamen}[6]{\fancyhead[R]{#1 (#2)\\#3 (#4)\\#5 (#6)}} %Kopfzeile rechts
\fancyhead[C]{\large{\bf{Blatt \vblatt }}}
\fancyhead[L]{\textbf{Datenstrukturen und Algorithmen}\\ Sommersemester 2017 \\ \"Ubungsgruppe: \vtermin%\\Briefkasten: \vbriefkasten 
}

\newcommand{\blatt}[1]{\def\vblatt{#1}}
\newcommand{\termin}[1]{\def\vtermin{#1}}
\newcommand{\briefkasten}[1]{\def\vbriefkasten{#1}}

%!!!!Neu Einfaches einbinden von Quelltexten
\usepackage{listings}

%Algorithmennotation als Quelltext verwenden.
\lstdefinelanguage{alglang}
{keywords={if, then, endif, else, repeat, until, endrepeat, while, do, endwhile, var, module, endmodule, mod, return, pre, post, reads, changes, mem, Type, Functions, Preconditions, Axioms},
emph={Integer, Bool, Real, Char, String},
sensitive=false,
morecomment=[l]{--},
morecomment=[s]{\{}{\}},
morestring=[d]"  
}
\lstdefinestyle{algstyle}{
  mathescape=true, 
  basicstyle=\normalfont\sffamily, 
  commentstyle=\normalfont\sffamily, 
  keywordstyle=\sffamily\bfseries,
  identifierstyle=\sffamily\itshape, 
  emphstyle=\normalfont\sffamily, 
  stringstyle=,
  showstringspaces=false,        
  columns=[l]fullflexible     
}

\lstdefinestyle{javastyle}{
	basicstyle=\ttfamily, 
  	commentstyle=\ttfamily, 
  	keywordstyle=\ttfamily,
  	identifierstyle=\ttfamily, 
  	emphstyle=\ttfamily,
%     frame=tb,
    language=Java,
    aboveskip=3mm,
    belowskip=3mm,
    showstringspaces=false,
    columns=flexible,
    basicstyle={\small\ttfamily},
    numbers=none,
    numberstyle=\tiny\color{gray},
    keywordstyle=\color{blue},
    commentstyle=\color{dkgreen},
    stringstyle=\color{mauve},
    breaklines=true,
    breakatwhitespace=true,
    tabsize=3
}

\newcommand{\lstjava}{\lstset { numbers=left,language=java,tabsize=2,numberstyle=\tiny ,numbersep=5pt,basicstyle=\scriptsize}}
\newcommand{\lstalg}{\lstset { numbers=left,language=alglang,tabsize=2,numberstyle=\tiny,style=algstyle}}

\newcounter{aufgabe}
\setcounter{aufgabe}{0}

\newcommand{\Aufgabe}{\noindent\newline\addtocounter{aufgabe}{1}\textbf{Aufgabe \vblatt.\theaufgabe}\\
%\input{\theaufgabe}\bigskip
}

% Unteraufgaben (mit Enumeration)
\def\labelenumi{(\arabic{enumi})}
\parindent0mm % keine Absatzeinrückung

% %%%%%%%%%%%%%%%%%%%%%%%%%%%%%%%%%%%%%%%%%%%%%%%%%%%%%%%%%%%%%%%%%%%%%%%%%%%% %
%                           Ende Header                                        %
% %%%%%%%%%%%%%%%%%%%%%%%%%%%%%%%%%%%%%%%%%%%%%%%%%%%%%%%%%%%%%%%%%%%%%%%%%%%% %

%!!!!anpassen an das Betriebssystem!!!, um Umlaute zu verwenden
% \usepackage[utf8]{inputenc}                      %Linux
%\usepackage[latin1]{inputenc}                    %Windows
%\usepackage[applemac]{inputenc}                  %Mac


%Namen und Matrikelnummern anpassen
\dreinamen{Ibraheem Saoud}{3041390}{Liang Chun Wong}{3037494}{Miroslav Valov}{3023792} %3er Gruppe


%Termin der Uebungsgruppe und Raum anpassen z. B. \termin{Mo 10:15-11:45, BB 915}
\termin{Di 16:15-18:45, BB 915}

%Blattnummer anpassen z. B. \blatt{1}
\blatt{8}


\begin{document}
\Aufgabe
\begin{enumerate}
\item Type
\subitem QUEUE[T]
\item Function
\subitem mt\_queue: QUEUE[T]
\subitem enqueue: T $\times$ QUEUE[T] $\rightarrow$ QUEUE[T]
\subitem dequeue: QUEUE[T] $\nrightarrow$ QUEUE[T]
\subitem head: QUEUE[T] $\nrightarrow$ T
\subitem empty: QUEUE[T] $\rightarrow$ Bool
\item Preconditions
\subitem $\forall$ q: QUEUE[T] $\bullet$
\subitem pre(dequeue(q)) $\Leftrightarrow$ empty(q) = false
\subitem pre(head(q)) $\Leftrightarrow$ empty(q) = false
\item Axioms
\subitem $\forall$ x, y : T; q: QUEUE[T] $\bullet$
\subitem dequeue(enqueue(x, mt\_queue)) = mt\_queue
\subitem dequeue(enqueue(x, enqueue(y, q))) = enqueue(x, dequeue(enqueue(y, q)))
\subitem head(enqueue(x, mt\_queue)) = x
\subitem head(enqueue(x, enqueue(y, q))) = head(enqueue(y, q))
\subitem empty(mt\_queue) = true
\subitem empty(enqueue(x, q)) = false
\end{enumerate}


\Aufgabe
\begin{lstlisting}[style=javastyle]
// Array.java 
public interface Array<T> {
    void put(int pos, T x);
    int lower();
    int upper();
    T get(int pos);
    boolean empty();
}

\end{lstlisting}
\newpage
\begin{lstlisting}[style=javastyle]
// MyArray.java 
import java.util.ArrayList;

public class MyArray<T> implements Array<T> {
    public ArrayList<T> arr;
    public int i;
    public int j;

    MyArray(int i, int j) {
        this.arr = new ArrayList<T>(j);
        for (int x = 0; x < i; x++) {
            this.arr.add(null);
        }
        for (int x = i; x <= j; x++) {
            this.arr.add(null);
        }
        this.i = i;
        this.j = j;
    }

    public int lower() {
        return this.i;
    }

    public int upper() {
        return this.j;
    }

    public T get(int pos) {
        if (pos < this.i) return null;
        return this.arr.get(pos);
    }

    public boolean empty() {
        for (int i = 0; i <= this.j; i++) {
          if(this.arr.get(i) != null)
              return false;
        }
        return true;
    }

    public void put(int i, T x) {
        if (i > this.j || i < this.i) return;
        this.arr.set(i, x);
    }

    public String toString() {
        ArrayList<T> clone = (ArrayList<T>)this.arr.clone();
        for (int i = this.i - 1; i >= 0; i--) {
            clone.remove(i);
        }
        return clone.toString();
    }
}

\end{lstlisting}
\newpage
\begin{lstlisting}[style=javastyle]
// Main.java
public class Main {
    public static void main(String args[]) {
        // Create Array[T] with type String and bounds 2 to 5
        MyArray<String> x = new MyArray<>(2, 5);

        // Insert two elements
        x.put(2, "ONE");
        x.put(3, "TWO");

        // Print the array
        System.out.println(x);

        // Check if it's empty
        System.out.println(x.empty());

        // Get element at position two
        System.out.println(x.get(2));

        // Get lower and upper bounds
        System.out.println("Lower bounds: " + x.lower());
        System.out.println("Upper bounds: " + x.upper());

        // Out of range put
        x.put(6, "INVALID ELEMENT");
        // Nothing will show in this println statement
        System.out.println(x);

    }
}
\end{lstlisting}

\newpage
\Aufgabe
\begin{lstlisting}[style=javastyle]
8.3.1 Bubble Sort:
start a: [3, 1, 6, 7, 2, 5, 4]
1. loop: [1, 3, 6, 2, 5, 4, 7]
2. loop: [1, 3, 2, 5, 4, 6, 7]
3. loop: [1, 2, 3, 4, 5, 6, 7]
4. loop: [1, 2, 3, 4, 5, 6, 7]

8.3.2 Insertion Sort:
start a: [3, 1, 6, 7, 2, 5, 4]
1. loop: [1, 3, 6, 7, 2, 5, 4]
2. loop: [1, 3, 6, 7, 2, 5, 4]
3. loop: [1, 3, 6, 7, 2, 5, 4]
4. loop: [1, 2, 3, 6, 7, 5, 4]
5. loop: [1, 2, 3, 5, 6, 7, 4]
6. loop: [1, 2, 3, 4, 5, 6, 7]

8.3.3 Selection Sort:
start a: [3, 1, 6, 7, 2, 5, 4]
1. loop: [1, 3, 6, 7, 2, 5, 4]
2. loop: [1, 2, 6, 7, 3, 5, 4]
3. loop: [1, 2, 3, 7, 6, 5, 4]
4. loop: [1, 2, 3, 4, 6, 5, 7]
5. loop: [1, 2, 3, 4, 5, 6, 7]
6. loop: [1, 2, 3, 4, 5, 6, 7]
\end{lstlisting}
\end{document}
